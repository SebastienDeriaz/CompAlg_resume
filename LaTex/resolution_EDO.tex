\documentclass[resume]{subfiles}


\begin{document}
\section{Résolution EDO}
$$\boxed{y'=f(t,y(t))}$$
\subsection{Méthodes explicites}
\subsubsection{Méthode d'Euler explicite}
$$y_{k+1}=y_k+hf(t_k,y_k)$$
$$\abs{\abs{\vec{y}_n-\vec{y}(T)}}\leq Ch$$
\begin{itemize}
\item 1 palier
\item ordre 2
\item Énergie totale non conservée
\item $h$ suffisamment petit si on veut être stable
\end{itemize}
\subsubsection{Méthode de Heun}
$$\vec{u}_{k+1}=\vec{u}_k+\frac{h}{2}\left(\vec{s}_1+\vec{s}_2\right)$$
$$\vec{s}_1=\vec{f}(t_k,\vec{u}_k)\qquad \vec{s}_2=\vec{f}(t_k+h,\vec{u}_k+h\vec{s}_1)$$

\begin{itemize}
\item 2 paliers
\item Ordre 2
\item $h$ suffisamment petit si on veut être stable
\end{itemize}
\begin{table}[H]
\centering
\begin{tabular}{c|cc}
$0$ & \\
$\frac{1}{2}$ & $\frac{1}{2}$\\\hline
 & $0$ & $1$
\end{tabular}
\end{table}
\subsubsection{Méthode optimale}
$$\boxed{\vec{u}_{k+1}=\vec{u}_k+h\left(\frac{1}{4}\vec{s}_1+\frac{3}{4}\vec{s}_2\right)}$$
\begin{align*}
\vec{s}_1&=f(t_k,\vec{u}_k)\\
\vec{s}_2&=f\left(t_k+\frac{2}{3}h,\vec{u}_k+\frac{2}{3}h\vec{s}_1\right)
\end{align*}
\begin{table}[H]
\centering
\begin{tabular}{c|cc}
0 &\\
$\frac{2}{3}$ & $\frac{2}{3}$\\\hline
 & $\frac{1}{4}$ & $\frac{3}{4}$
\end{tabular}
\end{table}
\subsubsection{Crank-Nicolson}
\begin{table}[H]
\centering
\begin{tabular}{c|cc}
0 & \\
1 & $1/2$ & $1/2$\\\hline
 & $1/2$ & $1/2$
\end{tabular}
\end{table}
\subsubsection{Runge-Kutta 2}
$$\boxed{\vec{u}_{k+1}=\vec{u}_k+h\left(w_1\vec{s}_1+w_2\vec{s}_2\right)}$$
$$\boxed{\vec{s}_1=\vec{f}(t_k,\vec{u}_k)\quad\vec{s}_2=\vec{f}(t_k+c_2h,\vec{u}_k+a_{21}h\vec{s}_1)}$$
\begin{table}[H]
\centering
\begin{tabular}{c|cc}
0 & \\
$c_2$ & $a_{21}$\\\hline
 & $w_1$ & $w_2$
\end{tabular}
\end{table}
\begin{enumerate}
\item Méthode à 2 paliers
\item Ordre 2
\item 
\end{enumerate}
\subsubsection{Runge-Kutta 3}
$$\boxed{\vec{u}_{k+1}=\vec{u}_k+h\left(w_1\vec{s}_1+w_2\vec{s}_2+w_3\vec{s}_3\right)}$$
\begin{align*}
\vec{s}_1&=\vec{f}(t_k,\vec{u}_k)\\
\vec{s}_2&=\vec{f}(t_k+c_2h,\vec{u}_k+a_{21}h\vec{s}_1)\\
\vec{s}_3&=\vec{f}(t_k+c_3h,\vec{u}_k+a_{31}h\vec{s}_1+a_{32}h\vec{s}_2)
\end{align*}
\begin{table}[H]
\centering
\begin{tabular}{c|ccc}
0 & \\
$c_2$ & $a_{21}$\\
$c_3$ & $a_{31}$ & $a_{32}$\\\hline
 & $w_1$ & $w_2$ & $w_3$
\end{tabular}
\end{table}
\subsubsection{Runge-Kutta 4}
$$\boxed{\vec{u}_{k+1}=\vec{u}_k+\frac{h}{6}\left(\vec{s}_1+2\vec{s}_2+2\vec{s}_3+\vec{s}_4\right)}$$
\begin{align*}
\vec{s}_1&=\vec{f}(t_k,\vec{u}_k)\\
\vec{s}_2&=\vec{f}\left(t_k+\frac{h}{2},\vec{u}_k+\frac{h}{2}\vec{s}_1\right)\\
\vec{s}_3&=\vec{f}\left(t_k+\frac{h}{2},\vec{u}_k+\frac{h}{2}\vec{s}_2\right)\\
\vec{s}_4&=\vec{f}\left(t_k+h,\vec{u}_k+h\vec{s}_3\right)
\end{align*}
\begin{table}[H]
\centering
\renewcommand{\arraystretch}{1.5}
\begin{tabular}{c|cccc}
0 & \\
$\frac{1}{2}$ & $\frac{1}{2}$\\
$\frac{1}{2}$ & 0 & $\frac{1}{2}$\\
1 & 0 & 0 & 1\\\hline
 & $\frac{1}{6}$ & $\frac{1}{3}$ & $\frac{1}{3}$ & $\frac{1}{6}$
\end{tabular}
\end{table}
\begin{enumerate}
\item Méthode à 4 paliers
\item Ordre 4
\item Conditionnellement stable
\end{enumerate}
\subsubsection{Stabilité}
Pour que le système soit stable (pour un système linéaire à une équation), il faut que
$$\boxed{h<\frac{2}{\lambda}}$$
Avec
$$\frac{du}{dt}=-\lambda u(t)$$
Pour un système de la forme
$$\frac{d\vec{u}}{dt}=-A\vec{u}$$
On va chercher le max des valeurs propres
$$\boxed{h<\frac{2}{\abs{\lambda_{max}(A)}}}$$
Pour un problème non-linéaire, on peut faire une linéarisation autour de $u_0$ et $t_0$
\begin{table}[H]
\centering
\renewcommand{\arraystretch}{1.5}
\begin{tabular}{c|cccc}
0 & 0\\\hline
 & 1
\end{tabular}
\end{table}
\subsection{Méthodes implicites}
\subsubsection{Méthode de Euler implicite}
Sois un système de la forme :
$$\frac{du}{dt}=f(t,u(t))$$
$$u(t_0)=u_0$$
La méthode d'Euler implicite est la suivante :
$$\frac{du}{dt}\Big|_{t_k}=f(t_k,u_k)$$
$$\frac{du}{dt}\Big|_{tk}\approx \frac{\vec{u}_k-\vec{u}_{k-1}}{h}$$
$$\boxed{\vec{u}_{k+1}=\vec{u}_{k}+hf(t_{k+1},\vec{u}_{k+1})}$$
\begin{table}[H]
\centering
\renewcommand{\arraystretch}{1.5}
\begin{tabular}{c|cccc}
1 & 1\\\hline
 & 1
\end{tabular}
\end{table}
\begin{enumerate}
\item Nécessité de résoudre un système linéaire à chaque fois
\item Ordre 1
\item Problèmes de conservation d'énergie
\item Inconditionnellement stable	
\end{enumerate}
\subsubsection{Méthode de Crank-Nicolson}
$$u_{k+1}=u_k+\int_{t_k}^{t_{k+1}}f(t,u(t))dt$$
On peut utiliser la méthode des trapèzes pour approximer l'intégrale
$$\approx \frac{h}{2}\left(f(t_k,u_k)+f(t_{k+1},u_{k+1})\right)$$
$$\boxed{\vec{u}_{k+1}=\vec{u}_k+\frac{h}{2}\left(\vec{f}(t_k,\vec{u}_k)+\vec{f}(t_{k+1},\vec{u}_{k+1})\right)}$$
On peut dire que Crank-Nicolson est la moyenne entre la méthode d'Euler explicite et implicite.
\begin{table}[H]
\centering
\renewcommand{\arraystretch}{1.5}
\begin{tabular}{c|cccc}
0 & \\
1 & $1/2$ & $1/2$\\\hline
 & $1/2$ & $1/2$
\end{tabular}
\end{table}
\begin{enumerate}
\item On doit résoudre un système linéaire à chaque fois
\item Ordre 2
\item Inconditionnellement stable
\end{enumerate}
\subsubsection{Méthode d'Euler symplectique}
Utilisé pour la mécanique
$$\boxed{\begin{pmatrix}
v_{k+1}\\x_{k+1}
\end{pmatrix}=\begin{pmatrix}
v_k+hF(t_k,x_k,v_k)\\
x_k+hv_{k+1}
\end{pmatrix}}$$






\subsection{Réduction d'ordre}
$$y(t)=y_1\quad y'(t)=y_2\quad \cdots \quad y^{(n+1)}=y_{n}$$
$$\vec{u}=\frac{d}{dt}\begin{pmatrix}
y_1\\
y_2\\
\vdots\\
y_n
\end{pmatrix}\qquad \vec{u}_0=\begin{pmatrix}
y(0)\\
y'(0)\\
\vdots\\
y^{(n+1)}(0)
\end{pmatrix}$$



\subsection{Tableau de Butcher}
\begin{table}[H]
\centering
\begin{tabular}{c|cccc}
$\textcolor{Orange}{c_1}$ & \\
$\textcolor{Orange}{c_2}$ & $\textcolor{RoyalBlue}{a_{21}}$\\
$\textcolor{Orange}{\vdots}$ & $\vdots$ & $\ddots$\\
$\textcolor{Orange}{c_n}$ & $\textcolor{OrangeRed}{a_{n1}}$ & $\cdots$ & $\textcolor{ForestGreen}{a_{n\ n-1}}$\\\hline
 & $\textcolor{Violet}{w_1}$ & $\textcolor{Violet}{\cdots}$ &$\textcolor{Violet}{w_{n-1}}$ &  $\textcolor{Violet}{w_n}$
\end{tabular}
\end{table}
$$\updownarrow$$
$$\vec{u}_{k+1}=\vec{u}_k+h\left(\textcolor{Violet}{w_1}\vec{s}_1+\cdots+\textcolor{Violet}{w_n}\vec{s}_n\right)$$
\begin{align*}
\vec{s}_1&=f(t_k+\textcolor{Orange}{c_1}h,\vec{u}_k)\\
\vec{s}_2&=f(t_k+\textcolor{Orange}{c_2}h,\vec{u}_k+h(\textcolor{RoyalBlue}{a_{21}}\vec{s}_1))\\
\vec{s}_n&=f(t_k+\textcolor{Orange}{c_n}h,\vec{u}_k+h\left(\textcolor{OrangeRed}{a_{n1}}\vec{s}_1+\cdots+\textcolor{ForestGreen}{a_{n\ n-1}}h\vec{s}_{n-1}\right))
\end{align*}
On doit avoir
\begin{align*}
w_1+w_2+\cdots+w_n&=1\\
a_{n1}+a_{n2}+\cdots+a_{nk}&=c_n
\end{align*}
\subsubsection{Erreur}
\begin{enumerate}
\item L'erreur globale (après toutes les itérations) est d'ordre égal à la taille du tableau
$$\abs{u_{\text{approché}}(T)-u_\text{réel}(T)}\leq Ch^n$$
\item L'erreur locale (commise après une itération) est d'ordre égal à la taille du tableau + 1
$$\abs{u_{\text{approché}}(T)-u_\text{réel}(T-h)}\leq Ch^{n+1}$$
\end{enumerate}    
\subsection{Problèmes mal posés}
Les problèmes mal posés ont une énorme perturbation de $u(t)$ pour une petite perturbation de $u_0$. Typiquement des soucis si on a une équation de la forme
$$\cdots (u_0-1)\cdots$$
Avec $u_0\approx 1$
\end{document}