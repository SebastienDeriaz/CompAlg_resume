\documentclass[resume]{subfiles}


\begin{document}
\section{Autres}
\subsection{Triangle de Pascal}
\begin{center}
\includegraphics[scale=1,page=1]{drwg_1.pdf}
\end{center}
$$(a+b)^3=a^3+3a^2b+3ab^2+b^3$$
$$(a+b)^n=\sum_{k=0}^{n}\begin{pmatrix}
n\\k
\end{pmatrix}x^ky^{n-k}$$

$$\begin{pmatrix}
n\\ k
\end{pmatrix}=C^{n}_{k}=\frac{n!}{k!(n-k)!}$$
\subsection{Matrices}
\subsubsection{Inverses}
$$\begin{pmatrix}
\textcolor{RoyalBlue}{a} & \textcolor{OrangeRed}{b} & \textcolor{ForestGreen}{c}\\
0 & \textcolor{Violet}{d} & \textcolor{Orange}{e}\\
0 & 0 & \textcolor{ProcessBlue}{f}
\end{pmatrix}^{-1}=\begin{pmatrix}
\frac{1}{\textcolor{RoyalBlue}{a}} & -\frac{\textcolor{OrangeRed}{b}}{\textcolor{RoyalBlue}{a}\textcolor{Violet}{d}} & \frac{\textcolor{OrangeRed}{b}\textcolor{Orange}{e}-\textcolor{ForestGreen}{c}\textcolor{Violet}{d}}{\textcolor{RoyalBlue}{a}\textcolor{Violet}{d}\textcolor{ProcessBlue}{f}}\\
0 & \frac{1}{\textcolor{Violet}{d}} & -\frac{\textcolor{Orange}{e}}{\textcolor{ProcessBlue}{f}\textcolor{Violet}{d}}\\
0 & 0 & \frac{1}{\textcolor{ProcessBlue}{f}}
\end{pmatrix}$$
Même principe si on renverse
$$\left(M^{T}\right)^{-1}=\left(M^{-1}\right)^{T}$$

$$\begin{pmatrix}
\textcolor{RoyalBlue}{a} & 0 & 0\\
\textcolor{OrangeRed}{b} & \textcolor{Violet}{d} & 0\\
\textcolor{ForestGreen}{c} & \textcolor{Orange}{e} & \textcolor{ProcessBlue}{f}
\end{pmatrix}^{-1}=\begin{pmatrix}
\frac{1}{\textcolor{RoyalBlue}{a}} & 0 & 0\\
-\frac{\textcolor{OrangeRed}{b}}{\textcolor{RoyalBlue}{a}\textcolor{Violet}{d}} & \frac{1}{\textcolor{Violet}{d}} & 0\\
\frac{\textcolor{OrangeRed}{b}\textcolor{Orange}{e}-\textcolor{ForestGreen}{c}\textcolor{Violet}{d}}{\textcolor{RoyalBlue}{a}\textcolor{Violet}{d}\textcolor{ProcessBlue}{f}} & -\frac{\textcolor{Orange}{e}}{\textcolor{ProcessBlue}{f}\textcolor{Violet}{d}} &
\frac{1}{\textcolor{ProcessBlue}{f}}
\end{pmatrix}$$
Pour une matrice $2\times 2$
$$\begin{pmatrix}
\textcolor{RoyalBlue}{a} & \textcolor{OrangeRed}{b}\\
\textcolor{ForestGreen}{c} & \textcolor{Violet}{d}
\end{pmatrix}^{-1}=\frac{1}{\textcolor{RoyalBlue}{a}\textcolor{Violet}{d}-\textcolor{OrangeRed}{b}\textcolor{ForestGreen}{c}}\begin{pmatrix}
\textcolor{Violet}{d} & \textcolor{OrangeRed}{-b}\\
\textcolor{ForestGreen}{-c} & \textcolor{RoyalBlue}{a}
\end{pmatrix}$$
\subsection{Méthode des moindres carrés (vu en SignProc)}
$$A^{+}=\left(A^HA\right)^{-1}A^H$$
$$\Phi=(A^TA)^{-1}A^Tb=A^{+}b$$
Avec la matrice $A$ sous la forme
$$A=\begin{pmatrix}
1 & x_1 & x_1^2\\
1 & x_2 & x_2^2\\
 & \vdots \\
1 & x_{n} & x_{n}^2
\end{pmatrix}$$
Autant de lignes que de points et autant de colonnes que de fonction de base.
\paragraph{Méthode 1}
$$A=\begin{bmatrix}
1 & x_i & y_i
\end{bmatrix}\qquad b=\begin{pmatrix}
0\\
\vdots\\
1
\end{pmatrix}$$
\paragraph{Méthode 2}
$$A=\begin{bmatrix}
1 & x_i
\end{bmatrix}\qquad b=\begin{bmatrix}
y_i
\end{bmatrix}$$



\subsection{A faire attention}
\begin{mdframed}[linewidth=2pt, linecolor=OrangeRed]
\begin{itemize}
\item Ne pas mélanger $F'$ et $F$ dans les résolution d'équation non linéaires
\item Ne pas mélanger $F$ et $f$ ! (point fixe ou autre méthode)
\end{itemize}
\end{mdframed}
\end{document}