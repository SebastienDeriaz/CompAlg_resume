\documentclass[resume]{subfiles}


\begin{document}
    \section{Résolution numérique de systèmes linéaires}
    \begin{align*}
    a_{11}x_1+\cdots+a_{1n}x_n &=b_1\\
    \vdots \\
    a_{n1}x_1 + \cdots + a_{nn}x_n &= b_n
    \end{align*}
    
    $$\underbrace{\begin{pmatrix}
    a_{11} & \cdots & a_{1n}\\
    \vdots & \ddots & \vdots\\
    a_{n1} & \cdots & a_{nn}
    \end{pmatrix}}_{A}\underbrace{\begin{pmatrix}
    x_1\\\vdots\\x_n
    \end{pmatrix}}_{\vec{x}}=\underbrace{\begin{pmatrix}
    b_1\\\vdots\\b_n
    \end{pmatrix}}_{\vec{b}}$$
	Si peu d'éléments sont non-nuls alors $A$ est dite \textbf{creuse}
	\subsection{Condition d'arrêt}
	$$\abs{\abs{\vec{r}_k}}=\abs{\abs{\vec{b}-A\vec{x}_k}}\leq \tau\abs{\abs{\vec{b}}}$$
	On peut aussi utiliser une condition d'arrêt sur l'erreur $\vec{e}_k$ au lieu du résidu $\vec{r}_k=\vec{b}-A\vec{x}_k$
	\subsubsection{Lien entre résidu et erreur}
	$$\frac{\abs{\abs{\vec{x}-\vec{x}_k}}}{\abs{\abs{\vec{x}_k}}_p}\leq \underbrace{\abs{\abs{A}}_p\abs{\abs{A^{-1}}}_p}_{\kappa_p(A)}\frac{\abs{\abs{\vec{b}-A\vec{x}_k}}_p}{\abs{\abs{\vec{b}}}_p}$$
	Autant de digits valides dans la mantisse que
	$$\abs{\log_{10}(\varepsilon)}-\log_{10}(\kappa(A)_p)$$
	Avec $\varepsilon$ la précision machine (1e-16 en général)
	\subsubsection{Perturbation}
	Perturbation sur $A$
	$$\boxed{\frac{\abs{\abs{\delta\vec{x}_A}}}{\abs{\abs{\vec{x}+\delta \vec{x}_A}}}\leq \abs{\abs{\vec{A}}}\cdot\abs{\abs{\vec{A}^{-1}}}\cdot\frac{\abs{\abs{\delta A}}}{\abs{\abs{A}}}}$$
	Perturbation sur $A$ et $\vec{b}$:
	$$\boxed{\frac{\abs{\abs{\delta\vec{x}}}}{\abs{\abs{\vec{x}+\delta \vec{x}}}}\leq \frac{\abs{\abs{\vec{A}}}\cdot\abs{\abs{\vec{A}^{-1}}}}{1-\abs{\abs{A}}\cdot \abs{\abs{A^{-1}}}\frac{\abs{\abs{\delta A}}}{\abs{\abs{A}}}}\cdot\left(\frac{\abs{\abs{\delta A}}}{\abs{\abs{A}}}+\frac{\abs{\abs{\delta \vec{b}}}}{\abs{\abs{\vec{b}}}}\right)}$$
	\subsection{Normes}
	$$\abs{\abs{\vec{v}}}_p=\left(\sum_{i=1}^{n}\abs{v_i}^p\right)^{\frac{1}{p}}$$
	\begin{enumerate}
	\item Vecteurs
		\begin{enumerate}
	\item 1-norme : somme des composantes
	\item 2-norme : norme euclidienne
	\item max-norme : $p\to\infty$
%	$$\abs{\abs{\vec{v}}}_{\infty}=\max(\abs{v_i})$$
	\end{enumerate}
	\item Matrices
	\begin{enumerate}
	\item 1-norme :
%	$$\abs{\abs{A}}_1=\max_{1\leq j\leq n}\sum_{i=1}^{n}\abs{a_{ij}}$$
	\item 2-norme : ou max des valeurs propres de $A^{T}A$
%	$$\abs{\abs{A}}_2=\sqrt{\max\abs{\lambda(A^{T}A)}}$$
	\item max-norme
%	$$\abs{\abs{A}}_{\infty}=\abs{\abs{A^{T}}}_1$$
	\end{enumerate}
	\end{enumerate}

	\begin{enumerate}
	\item $\abs{\abs{\vec{v}}}=0\longleftrightarrow \vec{v}=\vec{0}$
	\item $\abs{\abs{\lambda\vec{v}}}=\abs{\lambda}\cdot \abs{\abs{\vec{v}}}$
	\item $\abs{\abs{\vec{v}+\vec{u}}}\leq \abs{\abs{\vec{v}}}+\abs{\abs{\vec{u}}}$
	\end{enumerate}
	\subsection{Méthodes directes}
	\subsubsection{Élimination de Gauss sans pivot}
	Effectuer des combinaisons linéaires des lignes pour obtenir une matrice triangulaire supérieure. Matrice augmentée :
	$$\begin{amatrix}{3}
10 & -7 & 0 & 7\\
0 & -0.1 & 6 & 6.1\\
0 & 2.5 & 5 & 2.5\\
\end{amatrix}$$
\begin{enumerate}
\item Commencer par la colonne de gauche
\item Si nécessaire, permuter les lignes pour avoir un non-nul comme premier élément
\item Faire les combinaisons linéaires des lignes pour annuler les éléments inférieurs
\item Passer à la colonne suivante
\end{enumerate}

	
	
	
	
    


    
\end{document}