\documentclass[resume]{subfiles}


\begin{document}
    \section{Équations non linéaires}
	\subsection{Existence d'une solution}
	Solution $f(r)=0$ entre $a$ et $b$ pour $f$ continue
	$$f(a)\cdot f(b)<0$$
	\subsection{Bissection}
	\begin{enumerate}
	\item $a_0=a$, $b_0=b$, $x_0=\frac{a+b}{2}$
	\item Répéter jusqu'à ce que $\abs{a_k-b_k}\geq tol\cdot \abs{b_k}$
	\begin{enumerate}
	\item Si $f(x_k)f(a_k)<0$ : $a_{k+1}=a_k$, $b_{k+1}=x_k$
	\item Si $f(x_k)f(a_k)>0$ : $a_{k+1}=x_k$, $b_{k+1}=b_k$
	\item $x_{k}=\frac{a_{k}+b_{k}}{2}$
	\end{enumerate}
	\end{enumerate}
	L'erreur converge avec
	$$\abs{e_k}<\frac{b-a}{2^{k+1}}$$
	\begin{enumerate}
	\item Robuste
	\item Intervalles qui contiennent la solution
	\item Majorant connu de l'erreur
	\item Lente
	\end{enumerate}
	\subsection{Regula falsi}
A l'exception du premier terme, on doit décider si on utilise $x_{k-1}$ avec $x_{k-2}$ ou $x_{k-3}$
$$\boxed{x_{k}=\begin{cases}
x_{k-2}-y_{k-2}\frac{x_{k-1}-x_{k-2}}{y_{k-1}-y_{k-2}} & y_{k-1}\cdot y_{k-2} < 0\\
x_{k-3}-y_{k-3}\frac{x_{k-1}-x_{k-3}}{y_{k-1}-y_{k-3}} & y_{k-1}\cdot y_{k-3} < 0
\end{cases}}$$
\subsection{Sécante}
$$\boxed{x_{k+1}=x_{k}-f(x_k)\frac{x_k-x_{k-1}}{f(x_k)-f(x_{k-1})}}$$
Ordre de convergence de
$$p=\frac{1+\sqrt{5}}{2}=\phi$$
\begin{enumerate}
\item Pas de connaissance de la dérivée
\item Une seule évaluation de $f$
\item Plus efficace que Newton (plus facile à calculer)
\end{enumerate}
\subsection{Newton}
$$\boxed{x_{k+1}=x_{k}-\frac{f(x_{k}}{f'(x_{k})}}$$
$$\abs{e_{k+1}}\approx \abs{\frac{f''(r)}{f'(r)}}\cdot \abs{e_k}^2$$
\begin{enumerate}
\item Très rapide
\item Peut ne pas converger
\end{enumerate}
\subsection{Point fixe} 
On veut une équation sous la forme :
$$\boxed{x=F(x)\longrightarrow x_{k+1}=F(x_{k})}$$
Manipulation de $f$ pour obtenir $F$
$$f(x)=x^5+x^3-x-5\longrightarrow F(x)=x^5+x^3-5$$
$$C=\abs{F'(r)}$$
Pour gagner une décimale il faut que
$$k\geq \frac{\ln(10)}{\ln(C)}$$
\subsubsection{Théorème de Banach}
Si 
$$\abs{F(x_1)-F(x_2)}\leq L\abs{x_1-x_2}\qquad 0<L<1\quad \forall x_1,x_2\in I$$
Alors il existe une seule solution $r$ dans $I$
\begin{enumerate}
\item Fonction Lipschitzienne : qui satisfait la condition de Lipschitz avec n'importe quel $L$
\item Fonction contractante : qui satisfait la condition de Lipschitz et $0<L<1$
\end{enumerate}
$$L_\text{optimal}=\max_{x\in I}\abs{F'(x)}$$
\subsubsection{Estimation}
\begin{enumerate}
\item A priori : 
$$\abs{r-x_k}\leq \frac{L^{k}}{1-L}\abs{x_1-x_0}$$
\item A posteriori :
$$\abs{r-x_k}\leq \frac{L}{1-L}\abs{x_k-x_{k-1}}$$
\end{enumerate}




	


    
\end{document}